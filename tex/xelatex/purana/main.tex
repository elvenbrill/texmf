\documentclass[twoside]{purana}
\usepackage{purana}

\begin{document}

\frontmatter

\tableofcontents

\mainmatter

\chapter{Readable text}
Traditionally, text is \textsc{composed} to create a \emph{readable}, coherent, and visually \textbf{satisfying} typeface that works invisibly, without the awareness of the reader. Even distribution of typeset material, with a minimum of \dn{oṃ namaḥ śivāya oṃ} distractions and anomalies, is aimed at producing clarity and transparency. Choice of typeface(s) is the primary aspect of text typography—prose fiction, non-fiction, editorial, educational, religious, scientific, spiritual, and commercial writing all have differing characteristics and requirements of appropriate typefaces and their fonts or styles.

\begin{quote}
  \sa{mandākinīsalilacandanacarcitāya\\ nandīśvarapramathanāthamaheśvarāya\\ mandāramukhyabahupuṣpasupūjitāya\\ tasmai makārāya namaḥ śivāya}
\end{quote}

\dn{oṃ namaḥ śivāya oṃ} \eg 1\st 2\nd 3\rd 4\th \etc

Traditionally, text is composed to create a readable, coherent, and visually satisfying typeface that works invisibly, without the awareness of the reader. Even distribution of typeset material, with a minimum of distractions and anomalies, is aimed at producing clarity and transparency. Choice of typeface(s) is the primary aspect\footnote{\dn{oṃ namaḥ śivāya oṃ} \etal} of text typography—prose fiction, non-fiction, editorial, educational, religious, scientific, spiritual, and commercial writing all have differing characteristics and requirements of appropriate typefaces and their fonts or styles.

\begin{quote}
  \dn{vasiṣṭhakumbhodbhavagautamādi-\\
    munīndradevārcitaśekharāya\\
    candrārkavaiśvānaralocanāya\\
    tasmai vakārāya namaḥ śivāya (4)}
\end{quote}

Large shloka:

\largeshloka{
  मुद्रण कला मुद्रण को सजाने, मुद्रण डिज़ाइन तथा मुद्रण ग्लिफ्स को संशोधित करने की कला एवं तकनीक है। मुद्रण ग्लिफ़ को विभिन्न उदाहरण तकनीकों का उपयोग करके बनाया और संशोधित किया जाता
  तकनीक है। मुद्रण ग्लिफ़ को विभिन्न उदाहरण तकनीकों का
}

Това е текст на кирилица. Не е българска, но все пак е кирилица!

\begin{quote}
  \dn{
    मुद्रण कला मुद्रण को सजाने, मुद्रण डिज़ाइन तथा मुद्रण ग्लिफ्स को संशोधित करने की कला एवं तकनीक है। मुद्रण ग्लिफ़ को विभिन्न उदाहरण तकनीकों का उपयोग करके बनाया और संशोधित किया जाता
    तकनीक है। मुद्रण ग्लिफ़ को विभिन्न उदाहरण तकनीकों का
  }
\end{quote}

Traditionally, text is composed to create a readable, coherent, and visually satisfying typeface that works invisibly, without the awareness of the reader. Even distribution of typeset material, with a minimum of distractions and anomalies, is aimed at producing clarity and transparency. Choice of typeface(s) is the primary aspect of text typography—prose fiction, non-fiction, editorial, educational, religious, scientific, spiritual, and commercial writing all have differing characteristics and requirements of appropriate typefaces and their fonts or styles.

\dn{oṃ namo bhāgavate rūdrāya namaḥ śivāya namo namaḥ}

Traditionally, text is composed to create a readable, coherent, and visually satisfying typeface that works invisibly, without the awareness of the reader. Even distribution of typeset material, with a minimum of distractions and anomalies, is aimed at producing clarity and transparency. Choice of typeface(s) is the primary aspect of text typography—prose fiction, non-fiction, editorial, educational, religious, scientific, spiritual, and commercial writing all have differing characteristics and requirements of appropriate typefaces and their fonts or styles.

\begin{quote}
  \sa{oṃ namo bhāgavate rūdrāya namaḥ śivāya namo namaḥ oṃ namo bhāgavate rūdrāya namaḥ śivāya namo namaḥ oṃ namo bhāgavate rūdrāya namaḥ śivāya namo namaḥ oṃ namo bhāgavate rūdrāya namaḥ śivāya namo namaḥ oṃ namo bhāgavate rūdrāya namaḥ śivāya namo namaḥ}
\end{quote}

\chapter{Questions by the Sages}

O my Lord, Śrī Kṛṣṇa, son of Vasudeva, O all-pervading Personality of Godhead,
I offer my respectful obeisances unto You. I meditate upon Lord Śrī Kṛṣṇa
because He is the Absolute Truth and the primeval cause of all causes of
the creation, sustenance and destruction of the manifested universes. He is
directly and indirectly conscious of all manifestations, and He is independent
because there is no other cause beyond Him. It is He only who first imparted
the Vedic knowledge unto the heart of Brahmājī, the original living being.
By Him even the great sages and demigods are placed into illusion, as one is
bewildered by the illusory representations of water seen in fire, or land seen
on water. Only because of Him do the material universes, temporarily manifested
by the reactions of the three modes of nature, appear factual, although they are
unreal. I therefore meditate upon Him, Lord Śrī Kṛṣṇa, who is eternally existent
in the transcendental abode, which is forever free from the illusory
representations of the material world. I meditate upon Him, for He is
the Absolute Truth.

Completely rejecting all religious activities which are materially motivated,
this \emph{Bhāgavata-purāṇa} propounds the highest truth, which is
understandable by those devotees who are fully pure in heart. The highest truth
is reality distinguished from illusion for the welfare of all. Such truth
uproots the threefold miseries. This beautiful \emph{Bhāgavatam}, compiled by
the great sage Vyāsadeva [in his maturity], is sufficient in itself for God
realization. What is the need of any other scripture? As soon as one attentively
and submissively hears the message of \emph{Bhāgavatam}, by this culture of
knowledge the Supreme Lord is established within his heart.

O expert and thoughtful men, relish \emph{Śrīmad-Bhāgavatam}, the mature fruit
of the desire tree of Vedic literatures. It emanated from the lips of Śrī
Śukadeva Gosvāmī. Therefore this fruit has become even more tasteful, although
its nectarean juice was already relishable for all, including liberated souls.

Once, in a holy place in the forest of Naimiṣāraṇya, great sages headed by
the sage Śaunaka assembled to perform a great thousand-year sacrifice for
the satisfaction of the Lord and His devotees.

One day, after finishing their morning duties by burning a sacrificial fire and
offering a seat of esteem to Śrīla Sūta Gosvāmī, the great sages made inquiries,
with great respect, about the following matters.

The sages said: Respected Sūta Gosvāmī, you are completely free from all vice.
You are well versed in all the scriptures famous for religious life, and in
the Purāṇas and the histories as well, for you have gone through them under
proper guidance and have also explained them.

Being the eldest learned Vedāntist, O Sūta Gosvāmī, you are acquainted with
the knowledge of Vyāsadeva, who is the incarnation of Godhead, and you also know
other sages who are fully versed in all kinds of physical and metaphysical
knowledge.

And because you are submissive, your spiritual masters have endowed you with all
the favors bestowed upon a gentle disciple. Therefore you can tell us all that
you have scientifically learned from them.

Please, therefore, being blessed with many years, explain to us, in an easily
understandable way, what you have ascertained to be the absolute and ultimate
good for the people in general.

O learned one, in this iron Age of Kali men almost always have but short lives.
They are quarrelsome, lazy, misguided, unlucky and, above all, always disturbed.

There are many varieties of scriptures, and in all of them there are many
prescribed duties, which can be learned only after many years of study in their
various divisions. Therefore, O sage, please select the essence of all these
scriptures and explain it for the good of all living beings, that by such
instruction their hearts may be fully satisfied.

All blessings upon you, O Sūta Gosvāmī. You know for what purpose
the Personality of Godhead appeared in the womb of Devakī as the son of Vasudeva.

O Sūta Gosvāmī, we are eager to learn about the Personality of Godhead and His
incarnations. Please explain to us those teachings imparted by previous masters
  [\emph{ācāryas}], for one is uplifted both by speaking them and by hearing them.

Living beings who are entangled in the complicated meshes of birth and death can
be freed immediately by even unconsciously chanting the holy name of Kṛṣṇa,
which is feared by fear personified.

O Sūta, those great sages who have completely taken shelter of the lotus feet of
the Lord can at once sanctify those who come in touch with them, whereas
the waters of the Ganges can sanctify only after prolonged use.

Who is there, desiring deliverance from the vices of the age of quarrel, who is
not willing to hear the virtuous glories of the Lord?

His transcendental acts are magnificent and gracious, and great learned sages
like Nārada sing of them. Please, therefore, speak to us, who are eager to hear,
about the adventures He performs in His various incarnations.

O wise Sūta, please narrate to us the transcendental pastimes of the Supreme
Godhead’s multi-incarnations. Such auspicious adventures and pastimes of
the Lord, the supreme controller, are performed by His internal powers.

We never tire of hearing the transcendental pastimes of the Personality of
Godhead, who is glorified by hymns and prayers. Those who have developed a taste
for transcendental relationships with Him relish hearing of His pastimes at
every moment.

Lord Śrī Kṛṣṇa, along with Balarāma, played like a human being, and so masked He
performed many superhuman acts.

Knowing well that the Age of Kali has already begun, we are assembled here in
this holy place to hear at great length the transcendental message of Godhead
and in this way perform sacrifice.

We think that we have met Your Goodness by the will of providence, just so that
we may accept you as captain of the ship for those who desire to cross
the difficult ocean of Kali, which deteriorates all the good qualities of
a human being.

Since Śrī Kṛṣṇa, the Absolute Truth, the master of all mystic powers,
has departed for His own abode, please tell us to whom the religious principles
have now gone for shelter.

\chapter{Cañculā’s disillusion and detachment}

Śaunaka said: O Sūta of great intellect, thou art extremely blessed and
omniscient. By thy favour I am gratified to satiety again and again.

My mind rejoices much on hearing this old anecdote. Please narrate another story
equally increasing devotion to Śiva.

Nowhere in the world are those who drink nectar honoured with liberation. But in
regard to the nectar of the story of Śiva it is different. When drunk, it
straightway accords salvation.

Thou art blessed, blessed indeed. Blessed, blessed is the story of Śiva on
hearing which a man attains Śivaloka.

Śrī Sūta said: O Śaunaka, please listen I shall tell you, though it is a great
secret, since you are the foremost among Vedic scholars and a leading devotee
of Śiva.

There is a seaside village Bāṣkala where sinful people bereft of Vedic virtue
reside.

They are wicked debauchees with deceptive means of livelihood, atheists, farmers
bearing weapons and adulterous rogues.

They know not anything about true knowledge, detachment or true virtue. They are
brutish in their mental make-up and take a great deal of interest in listening
to evil gossips and slander.

People of different castes are equally roguish never paying attention to their
duties. Always drawn to worldly pleasures they are ever engrossed in one evil
action or another.

All the women too are equally crooked, whorish and sinful. Evil-tempered, loose
in morals they are devoid of good behaviour and disciplined life.

In the village Bāṣkala peopled by wicked people, there was a base
\emph{brāhmaṇa} called Binduga.

He was a wicked sinner traversing evil paths. Although he had a beautiful wife
he was enamoured of a prostitute. His passion for her completely upset his mind.

He forsook his devoted wife Cañculā and indulged in sexual dalliance with the
prostitute overwhelmed by Cupid’s arrows.

Many years thus elapsed without any abatement in his evil action. Afraid of
violating her chastity Cañculā, though smitten by Cupid bore her distress
  [calmly for a short while].

But later on as her youthful health and boisterous virility increased, Cupid’s
onslaught became extremely unbearable for her and she ceased from strictly
adhering to her virtuous conduct.

Unknown to her husband she began to indulge in sexual intercourse with her
sinful paramour at night. Fallen thus from sāttvic virtues she went ahead along
her evil ways.

O sage, once he saw his wife amorously indulging in sexual intercourse with her
paramour at night.

Seeing his wife thus defiled by the paramour at night he furiously rushed
at them.

When the roguish deceitful paramour knew that the wicked Binduga had returned to
the house he fled from the scene immediately.

The wicked Binduga caught hold of his wife and with threats and abuses fisted
her again and again.

The whorish wicked woman Cañculā thus beaten by her husband became infuriated
and spoke to her wicked husband.

Cañculā said: Foul-minded that you are, you indulge in sexual intercourse with
the harlot every day. You have discarded me your wife, ever ready to serve you
with my youthful body.

I am a youthful maiden endowed with beauty and mentally agitated by lust. Tell
me what other course can I take when I am denied the amorous sport with my
husband.

I am very beautiful and agitated with flush of fresh youth. Deprived of sexual
intercourse with you I am extremely distressed. How can I bear the pangs of
passion?

Sūta said: That base \emph{brāhmaṇa} Binduga, when addressed thus by his wife,
foolish and averse to his own duties said to her.

Binduga said: True indeed is what you have said with your mind agitated by
passion. Please listen, my dear wife, I shall tell you something that will be
of benefit to you. You need not be afraid.

You go ahead with your sexual sports with any number of paramours. No fear need
enter your mind. Extract as much of wealth as you can from them and give them
enough sexual pleasure.

You must hand over all the amount to me. You know that I am enamoured of my
concubine. Thus our mutual interests will be assured.

Sūta said: His wife Cañculā on hearing these words of her husband became
extremely delighted and assented to his vicious proposal.

Having thus entered into their nefarious mutual contract the two wicked
— husband and the wife — fearlessly went ahead with their evil actions.

A great deal of time was thus wasted by the foolish couple indulging in their
vicious activities.

The wicked Binduga, the \emph{brāhmaṇa} with a \emph{śūdra} woman for his
concubine, died after some years and fell into Hell.

The foolish fellow endured distress and torture in Hell for many days. He then
became a ghost in the Vindhya mountain range continuing to be terribly sinful.

After the death of her husband the wicked Binduga, the woman Cañculā continued
to stay in her house with her sons. The woman foolishly continued her amorous
dalliance with her paramours till she no longer retained her youthful charms.

Due to divine intercession it chanced that on an auspicious occasion she
happened to go to the Gokarṇa temple in the company of her kinsmen.

Casually moving about here and there with her kinsmen she happened to take her
bath in a holy pond as a normal routine affair.

In a certain temple a scholar of divine wisdom was conducting a discourse on
the holy \emph{Śiva-purāṇa} story some of which she happened to hear.

The portion that fell on her ears was the context in which it was said that
the servants of Yama would introduce a red hot iron into the vaginal passage of
women who indulge in sexual intercourse with their paramours. This narrative
made by the \emph{paurāṇika} to increase detachment, made the woman tremble
with fear.

At the end of the discourse when all the people dispersed, the terrified woman
approached the scholarly \emph{brāhmaṇa} and spoke to him in confidence.

Cañculā said: O noble sir, please listen to the ignoble activities which I
performed without knowing my real duties. O lord, on hearing the same you will
please take pity on me and lift me up.

O lord, with a mind utterly deluded I have committed very great sin. Blinded by
lust I spent the whole of my youth in incontinent prostitution.

Today on hearing your learned discourse abounding in the sentiments of
non-attachment I have become extremely terrified and I tremble much.

Fie upon me, the foolish sinner of a woman deluded by lust, censurable,
clinging to worldly pleasures and averse to my own duties.

Unknowingly a great sin that produces excessive distress has been committed by
me for a fleeting glimpse of an evanescent pleasure, a criminal action.

Alas, I do not know which terrible goal this will lead me to. My mind has always
been turned to evil ways. Which wise man will come to my succour there?

At the time of death how shall I face the terrible messengers of Yama? How shall
I feel when they tie nooses forcibly round my neck?

How shall I endure in Hell the mincing of my body to pieces? How shall I endure
the special torture that is excessively painful?

I bewail my lot. How can I peacefully proceed with the activity of my
sense-organs during the day? Agitated with misery how shall I get peaceful sleep
during the night?

Alas! I am undone! I am burnt down! My heart is torn to pieces! I am doomed in
every respect. I am a sinner of all sorts.

O adverse Fate! it was you who directed my mind along evil lines. With a hateful
stubbornness you made me commit great sins. I was led astray from the path of my
duty that would have bestowed all happiness.

O \emph{brāhmaṇa}, my present pain is millions of times more than that of a man
stuck to the stake or hurled from a high mountain-top.

My sin is so great that it cannot be washed away even if I take ablutions in
the Gaṅgā for a hundred years or even if I perform a hundred sacrifices.

What shall I do? Where shall I go? Whom shall I resort to? I am falling into
the ocean of Hell. Who can save me in this world?

O noble sir, thou art my preceptor. Thou art my mother. Thou art my father.
I seek refuge in Thee. I am in a pitiable plight. Lift me, lift me.

Śrī Sūta said: The intelligent \emph{brāhmaṇa} mercifully lifted up Cañculā who
had become disgusted [with worldly affairs] and had fallen at his feet.
That \emph{brāhmaṇa} then spoke [as follows].

\chapter{Śiva-pañcākṣara-stotram}

\begin{sanskrit}शिव-पञ्चाक्षर-स्तोत्रम्\end{sanskrit} | śiva-pañcākṣara-stotram

\begin{sanskrit}
  नागेन्द्रहाराय त्रिलोचनाय\\
  भस्माङ्गरागाय महेश्वराय ।\\
  नित्याय शुद्धाय दिगम्बराय\\
  तस्मै नकाराय नमः शिवाय ॥ १ ॥
\end{sanskrit}

nāgendrahārāya trilocanāya\\
bhasmāṅgarāgāya maheśvarāya\\
nityāya śuddhāya digambarāya\\
tasmai nakārāya namaḥ śivāya (1)

\begin{sanskrit}
  मन्दाकिनीसलिलचन्दनचर्चिताय\\
  नन्दीश्वरप्रमथनाथमहेश्वराय ।\\
  मन्दारमुख्यबहुपुष्पसुपूजिताय\\
  तस्मै मकाराय नमः शिवाय ॥ २ ॥
\end{sanskrit}

mandākinīsalilacandanacarcitāya\\
nandīśvarapramathanāthamaheśvarāya\\
mandāramukhyabahupuṣpasupūjitāya\\
tasmai makārāya namaḥ śivāya (2)

\begin{sanskrit}
  शिवाय गौरीवदनारविन्द-\\
  सूर्याय दक्षाध्वरनाशकाय ।\\
  श्रीनीलकण्ठाय वृषध्वजाय\\
  तस्मै शिकाराय नमः शिवाय ॥ ३ ॥
\end{sanskrit}

śivāya gaurīvadanāravinda-\\
sūryāya dakṣādhvaranāśakāya\\
śrīnīlakaṇṭhāya vṛṣadhvajāya\\
tasmai śikārāya namaḥ śivāya (3)

\begin{sanskrit}
  वसिष्ठकुम्भोद्भवगौतमादि-\\
  मुनीन्द्रदेवार्चितशेखराय ।\\
  चन्द्रार्कवैश्वानरलोचनाय\\
  तस्मै वकाराय नमः शिवाय ॥ ४ ॥
\end{sanskrit}

vasiṣṭhakumbhodbhavagautamādi-\\
munīndradevārcitaśekharāya\\
candrārkavaiśvānaralocanāya\\
tasmai vakārāya namaḥ śivāya (4)

\begin{sanskrit}
  यक्षस्वरूपाय जटाधराय\\
  पिनाकहस्ताय सनातनाय ।\\
  दिव्याय देवाय दिगम्बराय\\
  तस्मै यकाराय नमः शिवाय ॥ ५ ॥
\end{sanskrit}

yakṣasvarūpāya jaṭādharāya\\
pinākahastāya sanātanāya\\
divyāya devāya digambarāya\\
tasmai yakārāya namaḥ śivāya (5)

\begin{sanskrit}
  ॥ शिवपञ्चाक्षरस्तोत्रं सम्पूर्णम् ॥\\
\end{sanskrit}
śivapañcākṣarastotraṃ sampūrṇam

\end{document}
